\documentclass{article}

\usepackage{tikz,tkz-2d}
\usetikzlibrary{calc,intersections}

\pagestyle{empty}

\begin{document}
\begin{tikzpicture}
%----------------------------------------------------
% Coordinates of A, B and C, the triangle vertices 
%----------------------------------------------------
\coordinate[label=above:$A$] (A) at (5,4);
\coordinate[label=left:$B$] (B) at (0,0);
\coordinate[label=right:$C$] (C) at (7,0);

%----------------------------------------------------
% Lengths of segments [AB], [BC], and [CA] 
%----------------------------------------------------
\tkzMathLength(B,C)
\newlength{\la}
\setlength{\la}{\tkzMathLen pt}
\setlength{\la}{.01\la}
\tkzMathLength(A,C)
\newlength{\lb}
\setlength{\lb}{\tkzMathLen pt}
\setlength{\lb}{.01\lb}
\tkzMathLength(A,B)
\newlength{\lc}
\setlength{\lc}{\tkzMathLen pt}
\setlength{\lc}{.01\lc}

%----------------------------------------------------
% Computing 1/3 of each angle
%----------------------------------------------------
\pgfmathsetmacro{\A}{acos((\la*\la-\lb*\lb-\lc*\lc)/(-2*\lb*\lc))};
\pgfmathsetmacro{\tA}{\A/3};
\pgfmathsetmacro{\B}{acos((\lb*\lb-\la*\la-\lc*\lc)/(-2*\la*\lc))};
\pgfmathsetmacro{\tB}{\B/3};
\pgfmathsetmacro{\C}{acos((\lc*\lc-\lb*\lb-\la*\la)/(-2*\lb*\la))};
\pgfmathsetmacro{\tC}{\C/3};

%----------------------------------------------------
% Computing intersections of trisector lines
%----------------------------------------------------
\coordinate (A1) at ($(A)!100*max(\lb,\lc)!\tA:(B)$);
\coordinate (A2) at ($(A)!100*max(\lb,\lc)!2*\tA:(B)$);
\coordinate (B1) at ($(B)!100*max(\la,\lc)!\tB:(C)$);
\coordinate (B2) at ($(B)!100*max(\la,\lc)!2*\tB:(C)$);
\coordinate (C1) at ($(C)!100*max(\la,\lb)!\tC:(A)$);
\coordinate (C2) at ($(C)!100*max(\la,\lb)!2*\tC:(A)$);

%----------------------------------------------------
% Computing coordinates of vertices O, P and Q of 
% the Morley's triangle 
%----------------------------------------------------
\coordinate (O) at (intersection of C--C1 and A--A2);
\coordinate (P) at (intersection of A--A1 and B--B2);
\coordinate (Q) at (intersection of B--B1 and C--C2);

%----------------------------------------------------
% Drawing triangles and trisectors
%----------------------------------------------------
\tkzMarkAngle[size=1,fillcolor=green!80](B/A/A1)
\tkzMarkAngle[size=.9,fillcolor=green!80](A2/A/A1)
\tkzMarkAngle[size=.8,fillcolor=green!80](A2/A/C)
\tkzMarkAngle[size=1,fillcolor=blue!80](C/B/B1)
\tkzMarkAngle[size=.9,fillcolor=blue!80](B2/B/B1)
\tkzMarkAngle[size=.8,fillcolor=blue!80](B2/B/A)
\tkzMarkAngle[size=1,fillcolor=red!80](A/C/C1)
\tkzMarkAngle[size=.9,fillcolor=red!80](C2/C/C1)
\tkzMarkAngle[size=.8,fillcolor=red!80](C2/C/B)
\draw (A)--(B)--(C)--cycle;
\draw[fill=orange, opacity=.4] 
   (O)--node[sloped]{\tiny{//}}
   (P)--node[sloped]{\tiny{//}}
   (Q)--node[sloped]{\tiny{//}}(O);
\draw (A)--(O) (A)--(P) (B)--(P) (B)--(Q) (C)--(Q) (C)--(O);

%----------------------------------------------------
% Caption
%----------------------------------------------------
\node[rounded corners, fill=purple!20,anchor=south east] at (3,3) 
   {\begin{minipage}{5cm}
     \textbf{Morley's triangle}\newline In any triangle, trisector 
     lines intersect in 3 points that are vertices of an 
     equilateral triangle.
   \end{minipage}};

\end{tikzpicture}

\begin{tikzpicture}[scale=.8]
\tkzInit[ymin=-5,ymax=10]
\tkzClip[space=1]
\tkzPoint[pos=below left](0,0){A}
\tkzPoint(8,0){B}
\tkzMidPoint[color=red](A,B){O}
\tkzCircle*(A,B)
\tkzPoint[pos=above right](3.5,10){I}
\tkzSegment[lw=.5pt](I/A,I/B)
\tkzInterLCR(I,A)(O,4 cm){A}{M}
\tkzInterLCR(I,B)(O,4 cm){N}{B}
\tkzDrawPoint[mark={},pos=left](M)
\tkzDrawPoint[mark={},pos=right](N)
\tkzSegment(A/B,B/M,A/N)
\tkzRightAngle(A/M/B,A/N/B)
\tkzProjection(A,B)(I/J)
\tkzSegment[style=dashed,color=blue,lw=0.6pt](I/J)
\tkzInterLL[color=red,namecolor=blue](A,N)(B,M){H}
\end{tikzpicture}

\begin{tikzpicture}
\tkzInit
\tkzPoint[noname](0,0){C}
\tkzPoint[noname](4,0){A}
\tkzPoint[noname](0,3){B}
\tkzSquare*(B,A){E}{F}
\tkzSquare*(A,C){G}{H}
\tkzSquare*(C,B){I}{J}
\tkzFillPolygon[color = green ](A,C,G,H)
\tkzFillPolygon[color = blue ](C,B,I,J)
\tkzFillPolygon[color = purple](B,A,E,F)
\tkzPolygon[lw = 1pt](A,B,C)
\tkzPolygon[lw = 1pt](A,C,G,H)
\tkzPolygon[lw = 1pt](C,B,I,J)
\tkzPolygon[lw = 1pt](B,A,E,F)
\end{tikzpicture}

\begin{tikzpicture}
\tkzInit
\tkzPoint(0,0){A} \tkzPoint(3,3){B}
\tkzCircleR*[color=blue](B,B,A)
\tkzCircleR*[color=blue](A,A,B)
\tkzInterCC(B,B,A)(A,A,B){A2}{F}
\tkzCircleR*[color=red](F,A,B)
\tkzInterCC(F,A,B)(B,B,A){G}{A}
\tkzInterCC(F,A,B)(A,A,B){B}{E}
\tkzCircleR*[color=red](G,A,B)
\tkzInterCC(G,A,B)(B,B,A){A7}{A8}
\tkzCircleR*[color=red](A7,A,B)
\tkzInterCC(A7,A,B)(G,B,A){A9}{H}
\tkzDrawPoint[color=red](F,G,E,H)
\tkzInterLL(E,A)(H,B){O}
\tkzSegment[lw=.6 pt](A/B,O/H,O/E,E/H,F/O,G/O)
\end{tikzpicture}

\begin{tikzpicture}
\tkzInit \tkzClip[space=.5]
\tkzPoint[pos = below](0,0){C}\tkzPoint[pos = right](6,0){B}%
\tkzPoint[pos = above](3,8){A}
\tkzPolygon(A,B,C)%%
\tkzInCenter(A,B,C){I}
\tkzProjection(B,C)(I/H)
\tkzProjection[pos = left](A,C)(I/K)
\tkzProjection[pos = right](A,B)(I/L)
\tkzSegment(I/L,I/H,I/K)
\tkzCircle(I,H)
\tkzRightAngle(A/L/I,B/H/I,C/K/I)
\tkzSegmentMark[label = $r$,poslabel = -12pt](C/H)
\tkzSegmentMark[label = $a-r$,poslabel = -12pt](H/B)
\tkzSegmentMark[label = $r$](C/K)
\tkzSegmentMark[label = $b-r$](K/A)
\tkzSegmentMark[label = $a-r$](L/B)
\tkzSegmentMark[label = $b-r$](A/L)
\tkzSegmentMark[label = $r$](I/L)
\end{tikzpicture}

\newcommand*{\CircleInscribing}[2]{%
\xdef\ORadius{#1}
\xdef\OORadius{#2}
\pgfmathparse{(2*(\ORadius-\OORadius))/(\ORadius/\OORadius+1)}%
\global\let\OOORadius\pgfmathresult%
\pgfmathparse{\ORadius-\OOORadius}%
\global\let\OOOORadius\pgfmathresult%
\pgfmathparse{2*\OORadius-\ORadius}%
\global\let\XA\pgfmathresult%
\tkzPoint[pos = below left](0,0){O}
\ifdim\XA pt = 0pt\relax%
\tkzPoint[pos = below right](\XA,0){A}
\else%
\tkzPoint[pos = below left](\XA,0){A}
\fi%
\tkzPoint[pos = below left](\OORadius,0){D}
\tkzPoint[pos = below left](-\ORadius,0){X}
\tkzPoint[pos = below right](\ORadius,0){B}
\tkzPoint[name = $O_2$,pos = below left](\OORadius-\ORadius,0){O2}
\tkzSegmentMark[symbol = /](D/B,D/A)
\tkzCircleR(O,\ORadius)
\tkzCircleR(O2,\OORadius)
\tkzMediatorLine[prefix = m,kr = 2,kl = 0,style = dashed](A,B)
\tkzInterLCR(ml,mr)(O,\ORadius cm){C}{E}
\tkzLineOrth*[prefix = p](X,A)(A)
\ifdim\XA pt < 0 pt\relax%
\tkzInterLCR(pl,pr)(O,\OOOORadius cm){O4}{O3}
\else%
\tkzInterLCR(pr,pl)(O,\OOOORadius cm){O3}{O4}
\fi%
\tkzRightAngle(X/D/C,X/A/O3)
\tkzCircleR(O3,\OOORadius)
\tkzDrawPoint[name = $O_3$,pos = right](O3)
\tkzDrawPoint[pos = above right](C)
\tkzSegment[style = dashed](O/O3)
\tkzSegment(A/O3,C/B,C/A,X/B)
\tkzInterLCR(O,O3)(O,\ORadius cm){W}{Z}
\tkzSegment[label = $a$,time = .85,style = dashed](O/Z)
}%
\begin{tikzpicture}[scale = 1]
\tkzInit[xmin = -7,xmax = 7,ymin = -7,ymax = 7]
\tkzClip
\CircleInscribing{6}{4}
\end{tikzpicture}
\end{document}

